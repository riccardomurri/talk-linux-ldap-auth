\documentclass[english,serif,mathserif,usenames,dvipsnames]{beamer}
\usetheme[informal]{gc3}

\usepackage[T1]{fontenc}
\usepackage[utf8]{inputenc}
\usepackage{babel}

\usepackage{gc3}

\begin{document}

\title[LDAP auth]{Linux LDAP authentication}
\author{Riccardo Murri \texttt{<riccardo.murri@uzh.ch>}}
\date{\today}

%% Makes the title slide
\maketitle

\section{Intro}
\begin{frame}{What is LDAP?}

  \href{http://en.wikipedia.org/wiki/Ldap}{LDAP} is a distributed
  database, optimized for reading and searching.

  \+ These days, LDAP is mostly used for distributing
  \emph{authentication} information, i.e., as a replacement for
  \texttt{/etc/passwd}, NIS/YP, etc.
\end{frame}


\begin{frame}[fragile]
  \frametitle{Example LDIF entry}
\begin{semiverbatim}\scriptsize
\textbf{# ldapsearch -x -H ldaps://idldapmaster01.uzh.ch {\textbackslash}
  -W -D cn=idSysRoHPC2HPC,ou=Admin,ou=id,dc=uzh,dc=ch {\textbackslash}
  -b "ou=People,ou=HPC,ou=id,dc=uzh,dc=ch" "(uid=rmurri)"}
dn: uid=rmurri,ou=People,ou=HPC,ou=id,dc=uzh,dc=ch
objectClass: idAccount
objectClass: account
objectClass: posixAccount
objectClass: shadowAccount
objectClass: top
userPasswordHash: \{crypt\}DjY4deC6nz9F
uid: rmurri
cn: rmurri
homeDirectory: /home/user/rmurri
uidNumber: 6909
gidNumber: 1001
uzhuuid: u1042765
userPassword:: e2NyeXB0fUs0bG1vNnNHMnM5Q1U=
loginShell: /bin/bash
gecos: Riccardo Murri
mail: riccardo.murri@uzh.ch
\end{semiverbatim}
  \begin{seealso}
    \href{https://www.ietf.org/rfc/rfc2307.txt}{RFC 2307} for details.
  \end{seealso}
\end{frame}


\begin{frame}{Three options for LDAP auth in Linux}
  \begin{enumerate}
  \item
    \href{http://www.padl.com/OSS/pam_ldap.html}{libnss-ldap/pam-ldap}:
    Library-only implementation by PADL software
  \item \href{http://arthurdejong.org/nss-pam-ldapd/}{nss-pam-ldapd}:
    More recent implementation by A. De Jong,
    offloading LDAP connection to a shared service.
  \item
    \href{https://access.redhat.com/documentation/en-US/Red_Hat_Enterprise_Linux/6/html/Deployment_Guide/SSSD-Introduction.html}{SSSD}:
    Local authentication service with pluggable backends and caching
    features.
  \end{enumerate}
\end{frame}


\section{libnss-ldap / libpam-ldap}
\begin{frame}{libnss-ldap / libpam-ldap}
  \textbf{Pros:}
  \begin{itemize}
  \item very well known
  \item available for every distro
  \end{itemize}

  \+
  \textbf{Cons:}
  \begin{itemize}
  \item LDAP client library (and its dependencies) is loaded into \emph{every} running process
  \item potentially one LDAP connection per process and one LDAP search per \texttt{get*ent()} call
  \item requires \texttt{nscd} otherwise you're spamming the server
  \item login and other lookup operations can hang if network is not available
  \end{itemize}

  \begin{seealso}
    \url{https://wiki.debian.org/LDAP/NSS}
  \end{seealso}
\end{frame}

\begin{frame}[fragile]
  \frametitle{Example libnss-ldap configuration}
\begin{semiverbatim}\scriptsize
\textbf{# cat /etc/ldap.conf}
uri ldap://192.168.160.35 ldap://192.168.160.36
ldap_version 3
rootbinddn cn=root,dc=gc3,dc=uzh,dc=ch
pam_password md5
base dc=gc3,dc=uzh,dc=ch
nss_base_passwd        ou=People,dc=gc3,dc=uzh,dc=ch
nss_base_shadow        ou=People,dc=gc3,dc=uzh,dc=ch
nss_base_group         ou=Group,dc=gc3,dc=uzh,dc=ch
nss_initgroups_ignoreusers root,daemon,bin,...,postfix
\end{semiverbatim}
\end{frame}


\section{nss-pam-ldapd}
\begin{frame}
  \frametitle{nss-pam-ldapd}
  Tries to address some shortcomings of \texttt{libnss-ldap}.
  \begin{itemize}
  \item offloads all LDAP connection and searching to a separate daemon called \href{http://arthurdejong.org/nss-pam-ldapd/nslcd.8}{\texttt{nslcd}}
  \item NSS clients talk to \texttt{nslcd} over local UNIX domain sockets
  \end{itemize}

  \+
  \textbf{Pros:}
  \begin{itemize}
  \item simple and lightweight NSS client code
  \item \textbf{per host connection pooling}
  \item in case of network failures, \texttt{nslcd} fails quickly
  \item \textbf{flexible attribute mapping features}
  \end{itemize}

  \+
  \textbf{Cons:}
  \begin{itemize}
  \item code changes quickly: some features you need may not be available
    in the distro you're using
  \item superceded by SSSD: will likely remain a niche system
  \end{itemize}
\end{frame}


\begin{frame}[fragile]
  \frametitle{Example pam-nss-ldap configuration}
\begin{semiverbatim}\scriptsize
\textbf{# cat /etc/nslcd.conf}
uri ldap://192.168.160.35 ldap://192.168.160.36
ldap_version 3
uid nslcd
gid nslcd
base dc=gc3,dc=uzh,dc=ch
binddn cn=root,dc=gc3,dc=uzh,dc=ch
bindpw ********
ssl on
tls_reqcert demand
tls_cacertfile /etc/ssl/certs/ca-certificates.crt
nss_initgroups_ignoreusers root,daemon,bin,...,postfix
\end{semiverbatim}
\end{frame}


\section{SSSD}
\begin{frame}[fragile]
  \frametitle{SSSD, I}
  The \href{https://fedorahosted.org/sssd/}{\emph{System Security
      Services Daemon}} is developed by Red Hat and available in
  Fedora, RHEL6, but also Debian and Ubuntu.

  \+
  Same architecture/idea as\emph{nss-pam-ldapd}: separate client code
  into a daemon, and let NSS/PAM talk to it via a local UNIX-domain
  socket.

  \+
  Pluggable backends: SSSD starts one daemon per
  \emph{authentication domain}, so there is actually a family of
  processes:
\begin{semiverbatim}\tiny
# ps fauxwww | fgrep sss
root     Ss   Oct29   2:19 /usr/sbin/sssd -f -D
root     S    Oct29   8:21  \textbackslash\_ /usr/lib64/sssd/sssd_be -d 0 --debug-to-files --domain ldap1.ften.es.hpcn.uzh.ch
root     S    Oct29   0:28  \textbackslash\_ /usr/lib64/sssd/sssd_nss -d 0 --debug-to-files
root     S    Oct29   0:22  \textbackslash\_ /usr/lib64/sssd/sssd_pam -d 0 --debug-to-files
\end{semiverbatim}
\end{frame}


\begin{frame}
  \frametitle{SSSD, II}
  \textbf{Pros:}
  \begin{itemize}
  \item simple and lightweight NSS client code
  \item \textbf{per host connection pooling}
  \item \textbf{Not just for LDAP:} can do Active Directory, and Kerberos too
  \item \textbf{Can aggregate data from various sources}
  \item Can also retrieve and cache SSH \emph{known host} entries
  \end{itemize}

  \textbf{Cons:}
  \begin{itemize}
  \item Configuration is more complex
  \item Documentation is extensive and there currently is no concise overview
  \item Actively developed, some features you need may not be
    available in the distro you're using
  \item Own caching daemon distinct from \texttt{nscd}
  \end{itemize}
\end{frame}


\begin{frame}[fragile]
  \frametitle{Example SSSD configuration, I}
\begin{semiverbatim}\scriptsize
[sssd]
config_file_version = 2

\emph{# what services should SSSD interface to}
services = nss, pam

\emph{# index of auth backends}
domains = LDAP

[nss]
filter_users = root,ldap,\ldots,nscd

[pam]
reconnection_retries = 3
debug_level = 8
\end{semiverbatim}
\end{frame}

\begin{frame}[fragile]
  \frametitle{Example SSSD configuration, II}
\begin{semiverbatim}\scriptsize
\emph{# "Schroedinger" cluster users}
[domain/ldap1.ften.es.hpcn.uzh.ch]
id_provider = ldap
auth_provider = ldap
chpass_provider = ldap
access_provider = ldap
enumerate = true
cache_credentials = true

ldap_access_filter = uidNumber>1000

ldap_uri = ldap://ldap1.ften.es.hpcn.uzh.ch
ldap_tls_reqcert = never

ldap_schema = rfc2307
ldap_search_base = dc=unizh,dc=ch
ldap_user_search_base = ou=People,ou=Matterhorn,ou=zi,dc=unizh,dc=ch
ldap_group_search_base = ou=Groups,ou=Matterhorn,ou=zi,dc=unizh,dc=ch

ldap_default_bind_dn = cn=matterhorn,\ldots,ou=zi,dc=unizh,dc=ch
ldap_default_authtok = ********
\end{semiverbatim}
\end{frame}


\section{References}
\begin{frame}[fragile]
  \frametitle{Further reading}
  \footnotesize
  \begin{description}
  \item[RFC 2307, \url{https://www.ietf.org/rfc/rfc2307.txt}] standard LDAP schema for UNIX user/group information
  \item[RFC 2307bis, \url{https://tools.ietf.org/html/draft-howard-rfc2307bis-02}] draft update to RFC 2307
  \item[\url{https://wiki.debian.org/LDAP/NSS}] Debian's guide to configuring \texttt{libnss-ldap} and \texttt{libpam-ldap} (works almost verbatim in any Linux distro)
  \item[\url{http://arthurdejong.org/nss-pam-ldapd/}] home page of the \texttt{nss-pam-ldapd} project
  \item[\url{https://fedorahosted.org/sssd/}] home page of the SSSD project
  \item[\url{http://preview.tinyurl.com/rhel6-sssd-guide}] Red Hat's guide to installing and configuring SSSD
  \end{description}
\end{frame}


\appendix

\begin{frame}
  \frametitle{LDAP schemas}
  Entries in an LDAP database are sets of key/value pairs. (Keys need
  not be unique; equivalently: a key can map to multiple values.)

  \+
  An \emph{LDAP schema} specifies the names of allowed keys, and the type of
  corresponding values.

  \+
  \emph{Each entry} declares a set of schemas it conforms to; every
  attribute in an LDAP entry must be defined in some schema.
\end{frame}


\begin{frame}
  \frametitle{X.500/LDAP Directories}

  Entries are organized into a \emph{tree structure} (DIT).
  (So LDAP queries return subtrees, as opposed to flat sets of rows as
  in a RDBMS query.)

  \+
  Each entry is uniquely identified by a ``Distinguished Name'' (DN).
  The DN of an entry is formed by appending a one or more attribute
  values to the parent entry's DN.

  \+
  LDAP accesses might result in \emph{referrals}, which redirect the
  client to access another entry at a remote server.
\end{frame}

\end{document}

%%% Local Variables:
%%% mode: latex
%%% TeX-master: t
%%% End:
